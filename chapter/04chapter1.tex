% !TEX root = main.tex
\zihao{-4}\linespread{1.55}\selectfont
\chapter{模板使用说明}
\section{背景}
作者将要写毕业论文,发现学校并没有研究生毕业论文的\LaTeX 模板,因此根据\href{https://yjsxy.tyust.edu.cn/info/1172/3275.html}{《太原科技大学研究生学位论文格式的统一要求》}(以下简称《要求》)开发了一个研究生\LaTeX 毕业论文模板。TYUSTthesis(Taiyuan University of Science and Technology thesis)以\myverb{ctexbook}文档类为基础,以基本满足《要求》。但该模板并非官方模板,且不同学院或老师有不同要求,遇到问题请反馈。本模板部分格式参考
\href{https://github.com/tuxify/zzuthesis}{《郑州大学本科毕业设计(论文)和研究生学位论文(含 硕士和博士) LaTeX 模版》}

本项目Github 地址\faGithub :\href{https://github.com/Struggle-best/TYUST_thesis}{ \;\;https://github.com/Struggle-best/TYUST\_thesis}

邮箱:\href{fanchao11429@163.com}{fanchao11429@163.com}

本模板使用\TeX Live2022 + Xe\LaTeX 编译通过。\footnote{软件安装见:\url{http://tug.org/texlive/acquire.html}}
\section{文件结构}
本文档通过\textbf{main.tex} 文件\myverb{\input{ }}命令加入各个章节,\textbf{main.tex}内容如下,各部分按需加入自己文档。

\begin{lstlisting}
\begin{document}
% !TEX root = main.tex
%%%%%%%%%%%%%%%%%%%%%%%%%%%%%%%%%%%%%%%%%%%%%%%%%%
%% classn : 分类号
%% schoolc: 学校代码
%% sclass : 密级
%%%%%%%%%%%%%%%%%%%%%%%%%%%%%%%%%%%%%%%%%%%%%%%%%%
\classn{T P 3 9 1}%!必须敲空格实现俩端对齐!%
\schoolc{1 0 1 0 9}%!必须敲空格实现俩端对齐!%
\sclass{公开}
%%%%%%%%%%%%%%%%%%%%%%%%%%%%%%%%%%%%%%%%%%%%%%%%%%
%% ---------------- 论文封面信息 ----------------- 
%% ctitle     :中文题目
%% etitle     :英文题目
%% authors    :作者姓名
%% supervisor :导师及职称
%% department :培养单位
%% major      :学科专业
%% subdate    :论文提交日期
%% reportdate :论文答辩日期
%% chairman   :答辩委员会主席
%% type       :学术型 or 专业型 (硕士需要)
%%%%%%%%%%%%%%%%%%%%%%%%%%%%%%%%%%%%%%%%%%%%%%%%%%
\ctitle{太原科技大学硕士论文 \LaTeX 模板使用示例}
\etitle{Example of Using the Master's Thesis \LaTeX Template for Taiyuan University of Science and Technology}
\authors{樊超}
\supervisor{王帆 \;副教授}
\department{电子信息工程学院}
\major{控制科学与工程}
\subdate{2024年5月}
\reportdate{2024年5月1日}
\chairman{某某 教授}
\type{学术型}% 硕士论文需要这个信息
%%%%%%%%%%%%%%%%%%%%%%%%%%%%%%%%%%%%%%%%%%%%%%%%%%
%% ------------- 博士论文英文封面信息 ------------- 
%% eauthors     :作者姓名(英)
%% emajor       :学科专业(英)
%% edepartment  :培养单位(英)
%% eschool 		:学校名称(英)
%% esupervisor  :导师及职称(英)
%% edate        :论文提交日期
%%%%%%%%%%%%%%%%%%%%%%%%%%%%%%%%%%%%%%%%%%%%%%%%%%
\eauthors{FAN CHAO}
\emajor{Control Science and Engineering}
\edepartment{School of Electronic Infromation Engineering}
\eschool{Taiyuan University of Science and Technology}
\esupervisor{WANGFAN Associate Professor}
\edate{\today}%\edate{May 14,2024} %封面
\maketitle%封面
% ---------- 前文 --------- %
\frontmatter
% !TEX root = main.tex
\begin{cabstract}
中文摘要应简要说明本论文的目的、内容、方法、成果和结
论。要突出论文的创新之处。语言力求精炼、准确。中文摘要,
硕士学位论文要求字数为500 字左右,博士学位论文要求为
1000 字左右。中文摘要下方另起一行,注明本文的关键词(3-8
个)
\ckeywords{TeX/LaTeX;模板说明;关键词;3-8}

\end{cabstract}

\begin{eabstract}

The content of the English abstract should correspond closely to that of the Chinese abstract, while adhering to English grammar, maintaining clarity, and ensuring fluency of expression.
\ekeywords{TeX/LaTeX; Template Description; Keywords; 3-8}
\end{eabstract}


%摘要
\tableofcontents   %目录
%\listoffigures    %插图清单
%\listoftables     %表格清单
% !TEX root = main.tex
\begin{denotation}[2.5cm]
\item[t] 时间
\item[条目] \zhlipsum[1]
\end{denotation}%主要符号对照表

% ---------- 正文 --------- %
\mainmatter
% !TEX root = main.tex
\zihao{-4}\linespread{1.55}\selectfont
\chapter{模板使用说明}
\section{背景}
作者将要写毕业论文,发现学校并没有研究生毕业论文的\LaTeX 模板,因此根据\href{https://yjsxy.tyust.edu.cn/info/1172/3275.html}{《太原科技大学研究生学位论文格式的统一要求》}(以下简称《要求》)开发了一个研究生\LaTeX 毕业论文模板。TYUSTthesis(Taiyuan University of Science and Technology thesis)以\myverb{ctexbook}文档类为基础,以基本满足《要求》。但该模板并非官方模板,且不同学院或老师有不同要求,遇到问题请反馈。本模板部分格式参考
\href{https://github.com/tuxify/zzuthesis}{《郑州大学本科毕业设计(论文)和研究生学位论文(含 硕士和博士) LaTeX 模版》}

本项目Github 地址\faGithub :\href{https://github.com/Struggle-best/TYUST_thesis}{ \;\;https://github.com/Struggle-best/TYUST\_thesis}

邮箱:\href{fanchao11429@163.com}{fanchao11429@163.com}

本模板使用\TeX Live2022 + Xe\LaTeX 编译通过。\footnote{软件安装见:\url{http://tug.org/texlive/acquire.html}}
\section{文件结构}
本文档通过\textbf{main.tex} 文件\myverb{\input{ }}命令加入各个章节,\textbf{main.tex}内容如下,各部分按需加入自己文档。

\begin{lstlisting}
\begin{document}
% !TEX root = main.tex
%%%%%%%%%%%%%%%%%%%%%%%%%%%%%%%%%%%%%%%%%%%%%%%%%%
%% classn : 分类号
%% schoolc: 学校代码
%% sclass : 密级
%%%%%%%%%%%%%%%%%%%%%%%%%%%%%%%%%%%%%%%%%%%%%%%%%%
\classn{T P 3 9 1}%!必须敲空格实现俩端对齐!%
\schoolc{1 0 1 0 9}%!必须敲空格实现俩端对齐!%
\sclass{公开}
%%%%%%%%%%%%%%%%%%%%%%%%%%%%%%%%%%%%%%%%%%%%%%%%%%
%% ---------------- 论文封面信息 ----------------- 
%% ctitle     :中文题目
%% etitle     :英文题目
%% authors    :作者姓名
%% supervisor :导师及职称
%% department :培养单位
%% major      :学科专业
%% subdate    :论文提交日期
%% reportdate :论文答辩日期
%% chairman   :答辩委员会主席
%% type       :学术型 or 专业型 (硕士需要)
%%%%%%%%%%%%%%%%%%%%%%%%%%%%%%%%%%%%%%%%%%%%%%%%%%
\ctitle{太原科技大学硕士论文 \LaTeX 模板使用示例}
\etitle{Example of Using the Master's Thesis \LaTeX Template for Taiyuan University of Science and Technology}
\authors{樊超}
\supervisor{王帆 \;副教授}
\department{电子信息工程学院}
\major{控制科学与工程}
\subdate{2024年5月}
\reportdate{2024年5月1日}
\chairman{某某 教授}
\type{学术型}% 硕士论文需要这个信息
%%%%%%%%%%%%%%%%%%%%%%%%%%%%%%%%%%%%%%%%%%%%%%%%%%
%% ------------- 博士论文英文封面信息 ------------- 
%% eauthors     :作者姓名(英)
%% emajor       :学科专业(英)
%% edepartment  :培养单位(英)
%% eschool 		:学校名称(英)
%% esupervisor  :导师及职称(英)
%% edate        :论文提交日期
%%%%%%%%%%%%%%%%%%%%%%%%%%%%%%%%%%%%%%%%%%%%%%%%%%
\eauthors{FAN CHAO}
\emajor{Control Science and Engineering}
\edepartment{School of Electronic Infromation Engineering}
\eschool{Taiyuan University of Science and Technology}
\esupervisor{WANGFAN Associate Professor}
\edate{\today}%\edate{May 14,2024} %封面
\maketitle%封面
% ---------- 前文 --------- %
\frontmatter
% !TEX root = main.tex
\begin{cabstract}
中文摘要应简要说明本论文的目的、内容、方法、成果和结
论。要突出论文的创新之处。语言力求精炼、准确。中文摘要,
硕士学位论文要求字数为500 字左右,博士学位论文要求为
1000 字左右。中文摘要下方另起一行,注明本文的关键词(3-8
个)
\ckeywords{TeX/LaTeX;模板说明;关键词;3-8}

\end{cabstract}

\begin{eabstract}

The content of the English abstract should correspond closely to that of the Chinese abstract, while adhering to English grammar, maintaining clarity, and ensuring fluency of expression.
\ekeywords{TeX/LaTeX; Template Description; Keywords; 3-8}
\end{eabstract}


%摘要
\tableofcontents   %目录
%\listoffigures    %插图清单
%\listoftables     %表格清单
% !TEX root = main.tex
\begin{denotation}[2.5cm]
\item[t] 时间
\item[条目] \zhlipsum[1]
\end{denotation}%主要符号对照表

% ---------- 正文 --------- %
\mainmatter
% !TEX root = main.tex
\zihao{-4}\linespread{1.55}\selectfont
\chapter{模板使用说明}
\section{背景}
作者将要写毕业论文,发现学校并没有研究生毕业论文的\LaTeX 模板,因此根据\href{https://yjsxy.tyust.edu.cn/info/1172/3275.html}{《太原科技大学研究生学位论文格式的统一要求》}(以下简称《要求》)开发了一个研究生\LaTeX 毕业论文模板。TYUSTthesis(Taiyuan University of Science and Technology thesis)以\myverb{ctexbook}文档类为基础,以基本满足《要求》。但该模板并非官方模板,且不同学院或老师有不同要求,遇到问题请反馈。本模板部分格式参考
\href{https://github.com/tuxify/zzuthesis}{《郑州大学本科毕业设计(论文)和研究生学位论文(含 硕士和博士) LaTeX 模版》}

本项目Github 地址\faGithub :\href{https://github.com/Struggle-best/TYUST_thesis}{ \;\;https://github.com/Struggle-best/TYUST\_thesis}

邮箱:\href{fanchao11429@163.com}{fanchao11429@163.com}

本模板使用\TeX Live2022 + Xe\LaTeX 编译通过。\footnote{软件安装见:\url{http://tug.org/texlive/acquire.html}}
\section{文件结构}
本文档通过\textbf{main.tex} 文件\myverb{\input{ }}命令加入各个章节,\textbf{main.tex}内容如下,各部分按需加入自己文档。

\begin{lstlisting}
\begin{document}
% !TEX root = main.tex
%%%%%%%%%%%%%%%%%%%%%%%%%%%%%%%%%%%%%%%%%%%%%%%%%%
%% classn : 分类号
%% schoolc: 学校代码
%% sclass : 密级
%%%%%%%%%%%%%%%%%%%%%%%%%%%%%%%%%%%%%%%%%%%%%%%%%%
\classn{T P 3 9 1}%!必须敲空格实现俩端对齐!%
\schoolc{1 0 1 0 9}%!必须敲空格实现俩端对齐!%
\sclass{公开}
%%%%%%%%%%%%%%%%%%%%%%%%%%%%%%%%%%%%%%%%%%%%%%%%%%
%% ---------------- 论文封面信息 ----------------- 
%% ctitle     :中文题目
%% etitle     :英文题目
%% authors    :作者姓名
%% supervisor :导师及职称
%% department :培养单位
%% major      :学科专业
%% subdate    :论文提交日期
%% reportdate :论文答辩日期
%% chairman   :答辩委员会主席
%% type       :学术型 or 专业型 (硕士需要)
%%%%%%%%%%%%%%%%%%%%%%%%%%%%%%%%%%%%%%%%%%%%%%%%%%
\ctitle{太原科技大学硕士论文 \LaTeX 模板使用示例}
\etitle{Example of Using the Master's Thesis \LaTeX Template for Taiyuan University of Science and Technology}
\authors{樊超}
\supervisor{王帆 \;副教授}
\department{电子信息工程学院}
\major{控制科学与工程}
\subdate{2024年5月}
\reportdate{2024年5月1日}
\chairman{某某 教授}
\type{学术型}% 硕士论文需要这个信息
%%%%%%%%%%%%%%%%%%%%%%%%%%%%%%%%%%%%%%%%%%%%%%%%%%
%% ------------- 博士论文英文封面信息 ------------- 
%% eauthors     :作者姓名(英)
%% emajor       :学科专业(英)
%% edepartment  :培养单位(英)
%% eschool 		:学校名称(英)
%% esupervisor  :导师及职称(英)
%% edate        :论文提交日期
%%%%%%%%%%%%%%%%%%%%%%%%%%%%%%%%%%%%%%%%%%%%%%%%%%
\eauthors{FAN CHAO}
\emajor{Control Science and Engineering}
\edepartment{School of Electronic Infromation Engineering}
\eschool{Taiyuan University of Science and Technology}
\esupervisor{WANGFAN Associate Professor}
\edate{\today}%\edate{May 14,2024} %封面
\maketitle%封面
% ---------- 前文 --------- %
\frontmatter
% !TEX root = main.tex
\begin{cabstract}
中文摘要应简要说明本论文的目的、内容、方法、成果和结
论。要突出论文的创新之处。语言力求精炼、准确。中文摘要,
硕士学位论文要求字数为500 字左右,博士学位论文要求为
1000 字左右。中文摘要下方另起一行,注明本文的关键词(3-8
个)
\ckeywords{TeX/LaTeX;模板说明;关键词;3-8}

\end{cabstract}

\begin{eabstract}

The content of the English abstract should correspond closely to that of the Chinese abstract, while adhering to English grammar, maintaining clarity, and ensuring fluency of expression.
\ekeywords{TeX/LaTeX; Template Description; Keywords; 3-8}
\end{eabstract}


%摘要
\tableofcontents   %目录
%\listoffigures    %插图清单
%\listoftables     %表格清单
% !TEX root = main.tex
\begin{denotation}[2.5cm]
\item[t] 时间
\item[条目] \zhlipsum[1]
\end{denotation}%主要符号对照表

% ---------- 正文 --------- %
\mainmatter
% !TEX root = main.tex
\zihao{-4}\linespread{1.55}\selectfont
\chapter{模板使用说明}
\section{背景}
作者将要写毕业论文,发现学校并没有研究生毕业论文的\LaTeX 模板,因此根据\href{https://yjsxy.tyust.edu.cn/info/1172/3275.html}{《太原科技大学研究生学位论文格式的统一要求》}(以下简称《要求》)开发了一个研究生\LaTeX 毕业论文模板。TYUSTthesis(Taiyuan University of Science and Technology thesis)以\myverb{ctexbook}文档类为基础,以基本满足《要求》。但该模板并非官方模板,且不同学院或老师有不同要求,遇到问题请反馈。本模板部分格式参考
\href{https://github.com/tuxify/zzuthesis}{《郑州大学本科毕业设计(论文)和研究生学位论文(含 硕士和博士) LaTeX 模版》}

本项目Github 地址\faGithub :\href{https://github.com/Struggle-best/TYUST_thesis}{ \;\;https://github.com/Struggle-best/TYUST\_thesis}

邮箱:\href{fanchao11429@163.com}{fanchao11429@163.com}

本模板使用\TeX Live2022 + Xe\LaTeX 编译通过。\footnote{软件安装见:\url{http://tug.org/texlive/acquire.html}}
\section{文件结构}
本文档通过\textbf{main.tex} 文件\myverb{\input{ }}命令加入各个章节,\textbf{main.tex}内容如下,各部分按需加入自己文档。

\begin{lstlisting}
\begin{document}
\input{chapter/01coverinfor} %封面
\maketitle%封面
% ---------- 前文 --------- %
\frontmatter
\input{chapter/02abstract}%摘要
\tableofcontents   %目录
%\listoffigures    %插图清单
%\listoftables     %表格清单
\input{chapter/03denotation}%主要符号对照表

% ---------- 正文 --------- %
\mainmatter
\input{chapter/04chapter1}

\bibliography{chapter/13references.bib}%参考文献
% ---------- 附录 --------- %
%\begin{appendix}
%	\input{chapter/10appendices}
%\end{appendix}
\backmatter

\input{chapter/11achievement}%学术成果
\input{chapter/12acknowledgement}%致谢
\end{document}
\end{lstlisting}
\section{文件夹组成}
文件夹组成如下:

\begin{forest}
	pic dir tree,
	where level=0{}{directory,
	},
	[文件夹
	[chapter
	[01coverinfor.tex {\color{gray}论文封面信息},file2
	]
	[02abstract.tex {\color{gray}中英文摘要},file2
	]
	[03denotation.tex {\color{gray}主要符号对照表(可取消)},file2
	]
	[04chapter1.tex {\color{gray}第一章},file2
	]
	[10references.tex {\color{gray}参考文献},file2
	]
	[11appendices.tex {\color{gray}附录(可取消)},file2
	]
	[12achievement.tex {\color{gray}攻读学位期间取得的学术成果},file2
	]
	[13acknowledgement.tex {\color{gray}致谢},file2
	]
	]
	[figure
	[logo {\color{gray}封面logo}
	]
	[page2.png {\color{gray}论文图片存放}, file
	]
	[..., file
	]
	]
	[main.tex {\color{gray}主文件},file2
	]
	[TYUSTthesis.cls {\color{gray}模板格式文件},file2
	]
	]
\end{forest}

\section{参数说明}
本文档结合了硕士与博士学位论文模板,因此在加载TYUSTthesis文档类时有三个可选参数:
\begin{itemize}
	\item master 硕士学位论文(default)
	\item doctor 博士学位论文
	\item encover 博士学位论文英文封面
	\item declare 学位论文原创说明与授权说明
\end{itemize}

例如你想使用硕士论文并包含授权页,可使用下边命令
\begin{lstlisting}
\documentclass[master,declare]{TYUSTthesis}
\end{lstlisting}

你若想使用博士论文并包含英文封面,可使用下边命令
\begin{lstlisting}
\documentclass[doctor,encover]{TYUSTthesis}
\end{lstlisting}
\section{预加载宏包}
\begin{table}[hp]
	\renewcommand\arraystretch{1.2}
	\centering  % 显示位置为中间
	\bicaption{预加载宏包}{Preloaded macro package}\label{tab:1}
	\begin{tabular}{ll||ll} %第一列设置宽度为45pt 全为左对齐 没有分割线
		\hline
		宏包 		&说明		    & 宏包 	   & 说明     \\
		\hline 
		geometry  &页边距	     &gbt7714	  &参考文献格式 \\
		float	  &浮动体	     &setspace	  & 设置间距  \\
		mwe 	  &提供示例图片  & graphicx  & 插图       \\
		booktabs  &三线表       & longtable &长表格      \\
		makecell   &表格内容换行 & calc      & 距离计算   \\
		fontawesome& 字体       & multirow  & 合并单元格  \\
		mhchem     &化学环境     &chemfig   &化学环境     \\
		amsmath    &数学环境    &amsthm     &数学环境     \\
		amssymb    &数学环境    &amsfonts   &数学环境     \\
		hyperref   & 超链接    &enumitem    &列表         \\
		titletoc   & 目录设置  &setspace     &设置间距    \\
		caption    & 图表标题  &tikz        &画图         \\
		subcaption & 图表标题  &tcolorbox   & 彩色盒子   \\
		bicaption &  图表标题  &listings    & 代码环境    \\
		marginnote&  边注      &forest      &             \\
		xparse    &  边注      &zhlipsum    &             \\
        lipsum    &            &needspace  &              \\
		\hline
	\end{tabular}
\end{table}


\bibliography{chapter/13references.bib}%参考文献
% ---------- 附录 --------- %
%\begin{appendix}
%	\input{chapter/10appendices}
%\end{appendix}
\backmatter

\input{chapter/11achievement}%学术成果
% !TEX root = main.tex
\begin{ack}
\zhlipsum[1-2]



\end{ack}%致谢
\end{document}
\end{lstlisting}
\section{文件夹组成}
文件夹组成如下:

\begin{forest}
	pic dir tree,
	where level=0{}{directory,
	},
	[文件夹
	[chapter
	[01coverinfor.tex {\color{gray}论文封面信息},file2
	]
	[02abstract.tex {\color{gray}中英文摘要},file2
	]
	[03denotation.tex {\color{gray}主要符号对照表(可取消)},file2
	]
	[04chapter1.tex {\color{gray}第一章},file2
	]
	[10references.tex {\color{gray}参考文献},file2
	]
	[11appendices.tex {\color{gray}附录(可取消)},file2
	]
	[12achievement.tex {\color{gray}攻读学位期间取得的学术成果},file2
	]
	[13acknowledgement.tex {\color{gray}致谢},file2
	]
	]
	[figure
	[logo {\color{gray}封面logo}
	]
	[page2.png {\color{gray}论文图片存放}, file
	]
	[..., file
	]
	]
	[main.tex {\color{gray}主文件},file2
	]
	[TYUSTthesis.cls {\color{gray}模板格式文件},file2
	]
	]
\end{forest}

\section{参数说明}
本文档结合了硕士与博士学位论文模板,因此在加载TYUSTthesis文档类时有三个可选参数:
\begin{itemize}
	\item master 硕士学位论文(default)
	\item doctor 博士学位论文
	\item encover 博士学位论文英文封面
	\item declare 学位论文原创说明与授权说明
\end{itemize}

例如你想使用硕士论文并包含授权页,可使用下边命令
\begin{lstlisting}
\documentclass[master,declare]{TYUSTthesis}
\end{lstlisting}

你若想使用博士论文并包含英文封面,可使用下边命令
\begin{lstlisting}
\documentclass[doctor,encover]{TYUSTthesis}
\end{lstlisting}
\section{预加载宏包}
\begin{table}[hp]
	\renewcommand\arraystretch{1.2}
	\centering  % 显示位置为中间
	\bicaption{预加载宏包}{Preloaded macro package}\label{tab:1}
	\begin{tabular}{ll||ll} %第一列设置宽度为45pt 全为左对齐 没有分割线
		\hline
		宏包 		&说明		    & 宏包 	   & 说明     \\
		\hline 
		geometry  &页边距	     &gbt7714	  &参考文献格式 \\
		float	  &浮动体	     &setspace	  & 设置间距  \\
		mwe 	  &提供示例图片  & graphicx  & 插图       \\
		booktabs  &三线表       & longtable &长表格      \\
		makecell   &表格内容换行 & calc      & 距离计算   \\
		fontawesome& 字体       & multirow  & 合并单元格  \\
		mhchem     &化学环境     &chemfig   &化学环境     \\
		amsmath    &数学环境    &amsthm     &数学环境     \\
		amssymb    &数学环境    &amsfonts   &数学环境     \\
		hyperref   & 超链接    &enumitem    &列表         \\
		titletoc   & 目录设置  &setspace     &设置间距    \\
		caption    & 图表标题  &tikz        &画图         \\
		subcaption & 图表标题  &tcolorbox   & 彩色盒子   \\
		bicaption &  图表标题  &listings    & 代码环境    \\
		marginnote&  边注      &forest      &             \\
		xparse    &  边注      &zhlipsum    &             \\
        lipsum    &            &needspace  &              \\
		\hline
	\end{tabular}
\end{table}


\bibliography{chapter/13references.bib}%参考文献
% ---------- 附录 --------- %
%\begin{appendix}
%	\input{chapter/10appendices}
%\end{appendix}
\backmatter

\input{chapter/11achievement}%学术成果
% !TEX root = main.tex
\begin{ack}
\zhlipsum[1-2]



\end{ack}%致谢
\end{document}
\end{lstlisting}
\section{文件夹组成}
文件夹组成如下:

\begin{forest}
	pic dir tree,
	where level=0{}{directory,
	},
	[文件夹
	[chapter
	[01coverinfor.tex {\color{gray}论文封面信息},file2
	]
	[02abstract.tex {\color{gray}中英文摘要},file2
	]
	[03denotation.tex {\color{gray}主要符号对照表(可取消)},file2
	]
	[04chapter1.tex {\color{gray}第一章},file2
	]
	[10references.tex {\color{gray}参考文献},file2
	]
	[11appendices.tex {\color{gray}附录(可取消)},file2
	]
	[12achievement.tex {\color{gray}攻读学位期间取得的学术成果},file2
	]
	[13acknowledgement.tex {\color{gray}致谢},file2
	]
	]
	[figure
	[logo {\color{gray}封面logo}
	]
	[page2.png {\color{gray}论文图片存放}, file
	]
	[..., file
	]
	]
	[main.tex {\color{gray}主文件},file2
	]
	[TYUSTthesis.cls {\color{gray}模板格式文件},file2
	]
	]
\end{forest}

\section{参数说明}
本文档结合了硕士与博士学位论文模板,因此在加载TYUSTthesis文档类时有三个可选参数:
\begin{itemize}
	\item master 硕士学位论文(default)
	\item doctor 博士学位论文
	\item encover 博士学位论文英文封面
	\item declare 学位论文原创说明与授权说明
\end{itemize}

例如你想使用硕士论文并包含授权页,可使用下边命令
\begin{lstlisting}
\documentclass[master,declare]{TYUSTthesis}
\end{lstlisting}

你若想使用博士论文并包含英文封面,可使用下边命令
\begin{lstlisting}
\documentclass[doctor,encover]{TYUSTthesis}
\end{lstlisting}
\section{预加载宏包}
\begin{table}[hp]
	\renewcommand\arraystretch{1.2}
	\centering  % 显示位置为中间
	\bicaption{预加载宏包}{Preloaded macro package}\label{tab:1}
	\begin{tabular}{ll||ll} %第一列设置宽度为45pt 全为左对齐 没有分割线
		\hline
		宏包 		&说明		    & 宏包 	   & 说明     \\
		\hline 
		geometry  &页边距	     &gbt7714	  &参考文献格式 \\
		float	  &浮动体	     &setspace	  & 设置间距  \\
		mwe 	  &提供示例图片  & graphicx  & 插图       \\
		booktabs  &三线表       & longtable &长表格      \\
		makecell   &表格内容换行 & calc      & 距离计算   \\
		fontawesome& 字体       & multirow  & 合并单元格  \\
		mhchem     &化学环境     &chemfig   &化学环境     \\
		amsmath    &数学环境    &amsthm     &数学环境     \\
		amssymb    &数学环境    &amsfonts   &数学环境     \\
		hyperref   & 超链接    &enumitem    &列表         \\
		titletoc   & 目录设置  &setspace     &设置间距    \\
		caption    & 图表标题  &tikz        &画图         \\
		subcaption & 图表标题  &tcolorbox   & 彩色盒子   \\
		bicaption &  图表标题  &listings    & 代码环境    \\
		marginnote&  边注      &forest      &             \\
		xparse    &  边注      &zhlipsum    &             \\
        lipsum    &            &needspace  &              \\
		\hline
	\end{tabular}
\end{table}


\bibliography{chapter/13references.bib}%参考文献
% ---------- 附录 --------- %
%\begin{appendix}
%	\input{chapter/10appendices}
%\end{appendix}
\backmatter

\input{chapter/11achievement}%学术成果
% !TEX root = main.tex
\begin{ack}
\zhlipsum[1-2]



\end{ack}%致谢
\end{document}
\end{lstlisting}
\section{文件夹组成}
文件夹组成如下:

\begin{forest}
	pic dir tree,
	where level=0{}{directory,
	},
	[文件夹
	[chapter
	[01coverinfor.tex {\color{gray}论文封面信息},file2
	]
	[02abstract.tex {\color{gray}中英文摘要},file2
	]
	[03denotation.tex {\color{gray}主要符号对照表(可取消)},file2
	]
	[04chapter1.tex {\color{gray}第一章},file2
	]
	[10references.tex {\color{gray}参考文献},file2
	]
	[11appendices.tex {\color{gray}附录(可取消)},file2
	]
	[12achievement.tex {\color{gray}攻读学位期间取得的学术成果},file2
	]
	[13acknowledgement.tex {\color{gray}致谢},file2
	]
	]
	[figure
	[logo {\color{gray}封面logo}
	]
	[page2.png {\color{gray}论文图片存放}, file
	]
	[..., file
	]
	]
	[main.tex {\color{gray}主文件},file2
	]
	[TYUSTthesis.cls {\color{gray}模板格式文件},file2
	]
	]
\end{forest}

\section{参数说明}
本文档结合了硕士与博士学位论文模板,因此在加载TYUSTthesis文档类时有三个可选参数:
\begin{itemize}
	\item master 硕士学位论文(default)
	\item doctor 博士学位论文
	\item encover 博士学位论文英文封面
	\item declare 学位论文原创说明与授权说明
\end{itemize}

例如你想使用硕士论文并包含授权页,可使用下边命令
\begin{lstlisting}
\documentclass[master,declare]{TYUSTthesis}
\end{lstlisting}

你若想使用博士论文并包含英文封面,可使用下边命令
\begin{lstlisting}
\documentclass[doctor,encover]{TYUSTthesis}
\end{lstlisting}
\section{预加载宏包}
\begin{table}[hp]
	\renewcommand\arraystretch{1.2}
	\centering  % 显示位置为中间
	\bicaption{预加载宏包}{Preloaded macro package}\label{tab:1}
	\begin{tabular}{ll||ll} %第一列设置宽度为45pt 全为左对齐 没有分割线
		\hline
		宏包 		&说明		    & 宏包 	   & 说明     \\
		\hline 
		geometry  &页边距	     &gbt7714	  &参考文献格式 \\
		float	  &浮动体	     &setspace	  & 设置间距  \\
		mwe 	  &提供示例图片  & graphicx  & 插图       \\
		booktabs  &三线表       & longtable &长表格      \\
		makecell   &表格内容换行 & calc      & 距离计算   \\
		fontawesome& 字体       & multirow  & 合并单元格  \\
		mhchem     &化学环境     &chemfig   &化学环境     \\
		amsmath    &数学环境    &amsthm     &数学环境     \\
		amssymb    &数学环境    &amsfonts   &数学环境     \\
		hyperref   & 超链接    &enumitem    &列表         \\
		titletoc   & 目录设置  &setspace     &设置间距    \\
		caption    & 图表标题  &tikz        &画图         \\
		subcaption & 图表标题  &tcolorbox   & 彩色盒子   \\
		bicaption &  图表标题  &listings    & 代码环境    \\
		marginnote&  边注      &forest      &             \\
		xparse    &  边注      &zhlipsum    &             \\
        lipsum    &            &needspace  &              \\
		\hline
	\end{tabular}
\end{table}
